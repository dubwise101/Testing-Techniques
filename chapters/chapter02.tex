\section{Black-Box testing}
		
\subsection{Equivalence Partitioning}
	
Divide all possible inputs into classes (partitions) such that:
\begin{itemize}
	\item There is a finite number of input equivalence classes
	\item You may reasonably assume that
	\begin{itemize}
		\item the program behaves analogously for inputs in the same class
		\item one test with a representative value from a class is sufficient
		\item if the representative detects a fault
		then other class members would detect the same fault
	\end{itemize}
\end{itemize}
				
%TODO add image from slide
				
Strategy:
\begin{itemize}
	\item Identify input equivalence classes
	\begin{itemize}
		\item Based on conditions on inputs/outputs in specification/description
		\item Both valid and invalid input equivalence classes
		\item Based on heuristics and experience:
		\begin{itemize}
			\item
			\item
			\item
			\item$\ldots$
		\end{itemize}
	\end{itemize}
	\item Define one/couple of test cases for each class
	\begin{itemize}
		\item Test cases that cover valid classes
		\item Test cases that cover at most one invalid class
	\end{itemize}
\end{itemize}
				
%TODO add example
			
\subsection{Boundary Value Analysis}
		
\subsection{Error Guessing}
			
\begin{itemize}
	\item Just ‘guess’ where the errors are$\ldots$
	\item Intuition and experience of tester
	\item Ad hoc, not really a technique
	\item But can be quite effective
	\item Strategy:
	\begin{itemize}
		\item Make a list of possible errors or error-prone situations (often related to boundary conditions)
		\item Write test cases based on this list
	\end{itemize}
\end{itemize}
			
\section{White-Box Testing}
		
\begin{itemize*}
	\item Testing based on program code\\
	hence, programming language dependent
	\item Extent to which (source) code is executed, i.e.\ covered
	\item Different kind of coverage:
	\begin{itemize*}
		\item path coverage
		\item statement coverage
		\item (multiple-) condition coverage
		\item decision/branch coverage
		\item $\ldots$
	\end{itemize*}
\end{itemize*}
			
\subsection{Path Testing}
				
\begin{itemize}
	\item Execute every possible path of a program, i.e., every possible sequence of statements
	\item Strongest white-box criterion
	\item Usually impossible: infinitely many paths (in case of loops)
	\item So: not a realistic option
	\item But note: enormous reduction w.r.t.\ all possible test cases (each sequence of statements executed for only one value)
\end{itemize}
		
\subsection{Statement Coverage}
			
\begin{itemize*}
	\item Execute every statement of a program
	\item Relatively weak criterion 
	\item Weakest white-box criterion
\end{itemize*}
				
\subsection{Branch Coverage}
			
\begin{itemize}
	\item Bracn coverage == decision coverage
	\item Execute every branch of a program: each possible outcome of each decision occurs at least once
	\item Example:
	\begin{itemize}
		\item \verb|IF b THEN s1 ELSE s2|
		\item \verb|CASE x OF|
	\end{itemize}
\end{itemize}
			
\subsection{Condition Coverage}
			
\subsection{Branch Coverage}
